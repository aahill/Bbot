% Created 2015-09-11 Fri 13:09
\documentclass[aspectratio=169]{beamer}
\usepackage[utf8]{inputenc}
\usepackage[T1]{fontenc}
\usepackage{fixltx2e}
\usepackage{graphicx}
\usepackage{longtable}
\usepackage{float}
\usepackage{wrapfig}
\usepackage{rotating}
\usepackage[normalem]{ulem}
\usepackage{amsmath}
\usepackage{textcomp}
\usepackage{marvosym}
\usepackage{wasysym}
\usepackage{amssymb}
\usepackage{hyperref}
\tolerance=1000
\mode<beamer>{\usetheme{Madrid}}
\usetheme{default}
\author{Jake Brawer, Aaron Hill}
\date{\textit{<2015-08-10 Mon>}}
\title{Braitenbot Summer}
\hypersetup{
  pdfkeywords={},
  pdfsubject={},
  pdfcreator={Emacs 24.5.1 (Org mode 8.2.10)}}
\begin{document}

\maketitle
\begin{frame}{Outline}
\tableofcontents
\end{frame}

\section{What I Did This Summer}
\label{sec-1}
\begin{frame}[label=sec-1-1]{Important Summer Accomplishments}
\begin{itemize}
\item Added to, debugged, and validated code ( although this will be an ongoing process).
\begin{itemize}
\item Most important addition is an algorithm that automates the performance-wise ranking and crossing of a gen.
\end{itemize}
\item Developed tractable hypothesis(es) and corresponding experiment(s).
\item Ran and analyzed preliminary experiments (in simulation), confirming evolution is possible in our model, among other things.
\item Collected and organized code into a comprehensable document that will facilitate validation and future additions
\end{itemize}
\end{frame}

\begin{frame}[label=sec-1-2]{Crossing Algorithm}
Designed an algorithm that crosses organisms based on task performance.
\begin{itemize}
\item Organisms are placed in pools of two with the top two performers in the highest pool, the next two in the second highest, etc.
\item After orgs are crossed (starting with the highets pool) they are placed in successive pools for more chances to reproduce.
\begin{itemize}
\item Result is top perfromers mate multiple times with potentially many different orgs
\end{itemize}
\end{itemize}
\end{frame}

\begin{frame}[label=sec-1-3]{Hypothesis}
Given the mutability of the distribution and number of crossover points, we propose the following hypothesis:\\

\begin{block}{H$_{\text{1}}$}
The distribution of crossover points influences the robustness of an organism's traits, and thus, is directly related to evolvability, fitness, and modualrity.\\
\end{block}
\end{frame}

\begin{frame}[label=sec-1-4]{Experiments}
I ran multiple simulated populations in order to verify that the model is sensitive to selection pressures.
\begin{itemize}
\item Selected for number of 'functional' threads.
\item Altered a number of genomic parameters across populations to determine how they would affect selection.
\end{itemize}
\end{frame}

\begin{frame}[label=sec-1-5]{Results}
\includegraphics[width=12.0 cm,height=7.5 cm]{/home/jake/org/selection-comparison-1.png}
\end{frame}

\begin{frame}[label=sec-1-6]{Take Home}
\begin{itemize}
\item Adaptive evolution is possible
\item Due to many factors, including small population size,there is not much variation in the population
\begin{itemize}
\item This is mitigated in part by a very high mutation rate (An order of magnitude larger than the default)
\end{itemize}
\item Very high mutation rates are required because of the large amount of noncoding DNA
\begin{itemize}
\item Likewise, more adding more crossover points did not affect performance significantly due to being distributed amongst noncoding regions.
\end{itemize}
\end{itemize}

\begin{block}{}
Possible experiment/I.V.: alter noncoding region size and see how fitness modularity are affected. 
\end{block}
\end{frame}
\section{Next steps}
\label{sec-2}
\begin{frame}[label=sec-2-1]{Next Steps}
\begin{itemize}
\item Firm up experimental parameters 
\begin{itemize}
\item What does it mean for an org to be viable?
\item Max number of threads?
\end{itemize}
\item Run populations in simulation with smaller non-coding regions (Is this more or less adaptive?)
\item Complete/tweak experimental protocol
\item Run experiments
\end{itemize}
\end{frame}
% Emacs 24.5.1 (Org mode 8.2.10)
\end{document}